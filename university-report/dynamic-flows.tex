\subsection{Dynamic Flows}\label{subsec:dynamic-flows}

We begin by defining the model of dynamic flows, also called flows over time.
Here, Vickrey's fluid queuing model is used. \todo{Add cite}
The traffic network is represented as a directed Graph $(V, E)$ with a finite set of nodes $V$ and a finite set of edges $E$.
Although, we allow parallel edges, we often write $e=vw\in E$ for a directed edge $e$ from node $v$ to node $w$.
For a node $v$, we denote the set of \emph{outgoing edges} of $v$ as $\outEdges{v}\coloneqq \{ vw \in E \}$ and the set of \emph{incoming edges} as $\inEdges{v} \coloneqq \{ uv \in E \}$.

Each edge $e$ has a positive \emph{rate capacity} $\capa_e > 0$ and a positive \emph{transit time} $\transit_e > 0$.
The rate capacity bounds the amount of flow an edge can transfer at a time, which can be imagined as the width of a conveyor belt.
The transit time on the other is the time the conveyor belt needs to transfer particles from its beginning to its end.

We often analyze multi-commodity flows.
That means, we have a finite set $I$ of commodities and each commodity $i\in I$ has a \emph{source node} $s_i\in V$ and a \emph{sink node} $t_i\in V\setminus\{ s_i \}$.
We require, that $s_i$ can reach $t_i$ in $(V,E)$.
Furthermore, we define $V_i \coloneqq \{ v \in V \mid \text{$s_i$ can reach $v$ and $v$ can reach $t_i$ in (V, E)} \}$ as the subset of nodes that are relevant to commodity $i$.

Moreover, the \emph{network inflow rate} of a commodity is given by a locally integrable function $u_i:\R \rightarrow \R_{\geq 0}$ with $u_i(\theta) = 0$ for $\theta \leq 0$.

\begin{definition}
    A \emph{(dynamic) flow} is a pair $f = (\infl, \outfl)$ of families of locally integrable functions with $\infl_{i,e}, \outfl_{i,e} : \R \rightarrow \R_{\geq 0}$ with $\infl_{i,e}(\theta) = \outfl_{i,e}(\theta) = 0$ for $\theta \leq 0$ and  $e\in E$ and $i\in I$.
    Here, $\infl_{i,e}(\theta)$ is called the \emph{inflow rate} of commodity $i$ into edge $e$ at time $\theta$ whereas $\outfl_{i,e}(\theta)$ describes the \emph{outflow rate} of commodity $i$ out of $e$ at time $\theta$.
    We denote the \emph{total inflow and outflow rates} of an edge $e$ as $\infl_e(\theta) \coloneqq \sum_{i\in I} \infl_{i,e}(\theta)$ and $\outfl_e(\theta) \coloneqq \sum_{i\in I}\outfl_{i,e}(\theta)$.

    Given a dynamic flow, we can define the \emph{accumulative inflow and outflow} as
    \[
        F^+_e(\theta) \coloneqq \int_0^\theta \infl_e(t) \diff t \quad\text{and}\quad  F^-_e(\theta) \coloneqq \int_0^\theta \outfl_e(t) \diff t,
    \]
    respectively.
    Based on that, the \emph{queue length} of an edge $e$ at time $\theta$ can be determined as $\qulen_e(\theta) \coloneqq F^+_e(\theta) - F^-_e(\theta + \transit_e)$.
    The time-dependent cost of traversing an edge is thus defined as $c_e(\theta) \coloneqq \transit_e + \qulen_e(\theta) / \capa_e$.
\end{definition}

This definition already leads to a good insight on how particles should behave in a dynamic flow of this type:
A particle of commodity $i$ is generated at the source $s_i$ at some time $\theta$ and immediately decides which outgoing edge $e=s_i v\in \outEdges{s_i}$ it should take.
At this time, the cost of $e$ is $c_e(\theta) = \transit_e + \qulen_e(\theta) / \capa_e$, which means that the particle has to queue for $\qulen_e(\theta) / \capa_e$ time units before it can traverse the edge in $\transit_e$ time.
It will thus arrive at $v$ at time $T_e(\theta) \coloneqq \theta + c_e(\theta)$.
Once it arrives at an intermediary node $v\neq t_i$, it will again have an opportunity to update its routing decision by choosing a new outgoing edge in~$\outEdges{v}$ unless it arrives at its destination $t_i$. 

To make the dynamic flows behave like imagined above, we have to make several restrictions forming the class of feasible dynamic flows:

\begin{definition}\label{def:feasible-flow}
    A dynamic flow $f$ is called \emph{feasible up to time $H\in\R \cup \{\infty \}$}, if it fulfills the following conditions for almost all $\theta < H$:
    \begin{enumerate}[label=(F\arabic*)]
        \item\label{def:feasible-flow:max-capacity} No total outflow rate exceeds the capacity, i.e. $\forall e\in E: \outfl_e(\theta)\leq \capa_e$.
        \item\label{def:feasible-flow:deterministic-split} Flow traverses an edge in a FIFO-manner, i.e. \[
            \forall e\in E: \quad 
            \outfl_{i,e}(\theta) = \begin{cases}
                \outfl_e(\theta) \cdot \frac{\infl_{i,e}(\xi)}{\infl_e(\xi)}, & \text{if $\infl_e(\xi) > 0$,}\\
                0, & \text{otherwise,}
        \end{cases}
        \]
        where $\xi\coloneqq \max \{ \xi  \mid T_e(\xi) = \theta \}$ is the latest departure time a particle can arrive at $w$ at time $\theta$ by traversing edge $e=vw$. 
        \item\label{def:feasible-flow:flow-conservation} Flow conservation is preserved on intermediary nodes, i.e. \[\forall i\in I, v\in V\setminus\{ s_i, t_i \}: \quad
        \bal{i}{v}(\theta)\coloneqq \sum_{e\in\outEdges{v}}\infl_{i,e}(\theta) - \sum_{e\in\inEdges{v}} \outfl_{i,e}(\theta) = 0 \]
        and $\bal{i}{s_i}(\theta) = u_i(\theta)$ as well as $\bal{i}{t_i}(\theta) \leq 0$ for all $\theta\in\R$.

        \item\label{def:feasible-flow:operate-at-capacity} Edges operate at capacity, i.e. 
        \[
            \forall e\in E: \quad \outfl_{e}(\theta) = \begin{cases}
                \capa_e, &\text{if $\qulen_e(\theta - \transit_e)>0$,} \\
                \infl_e(\theta-\transit_e), &\text{otherwise.}
            \end{cases}
        \]
    \end{enumerate}
    A feasible flow up to $\infty$ is also called a \emph{feasible} flow.
\end{definition}



\begin{remark}
    We highlight, that property~\ref{def:feasible-flow:deterministic-split} is defined slightly different in literature such as~\cite{}
\todo{
    Remark, that we switched min with max
}
\end{remark}

We note, that for a given feasible flow $f$ the costs $c_e$ follow the FIFO-rule and hence the results from Section~\ref{sec:fifo-costs} can be applied.
\begin{proposition}
    The cost functions $c$ induced by a feasible flow $f$ follow the FIFO-rule.
\end{proposition}
\begin{proof}
    This can be seen by applying property~\ref{def:feasible-flow:max-capacity} in the following inequality for $\theta_1\leq \theta_2$:
    \begin{align*}
        T_e(\theta_2) - T_e(\theta_1)
        &= \theta_2 - \theta_1 + \frac{q_e(\theta_2) - q_e(\theta_1)}{\capa_e}
        = \theta_2 - \theta_1 + \frac{\int_{\theta_1}^{\theta_2}f_e^+(t)\diff t - \int_{\theta_1 + \transit_e}^{\theta_2 + \transit_e} f_e^-(t) \diff t}{\capa_e}\\
        &\geq \theta_2 - \theta_1 + \frac{\int_{\theta_1}^{\theta_2} \infl_e(t)\diff t - (\theta_2 - \theta_1)\cdot \capa_e}{\capa_e} \geq 0.
    \end{align*}
\end{proof}
