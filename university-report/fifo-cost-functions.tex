\section{Time-Dependent FIFO Cost Functions}\label{sec:fifo-costs}

Time-dependent cost functions that follow the First In First Out (FIFO) rule are very important for the algorithms introduced later.
We begin by defining the FIFO rule in a directed graph $(V, E)$.


\begin{definition}
    A time-dependent cost function $c: \R\rightarrow \R_{\geq0}^E$ \emph{follows the FIFO rule} if for all $e\in E$ the function $T_e: \theta\mapsto \theta + c_e(\theta)$ is monotone increasing.
    It \emph{follows the strong FIFO rule} if $T_e$ is strictly increasing for all $e\in E$.
    The function $T_e$ is called \emph{traversal time} of $e$.
\end{definition}

For the following we concentrate on a graph $(V,E)$ where all nodes can reach a specific sink node $t\in V$.
Moreover, let $c:\R\rightarrow \R_{\geq0}^E$ be a time-dependent cost function.

The traversal time $T_P$ of a path $P = e_1\,\cdots\,e_k$ is given by the concatenation of the edges' traversal times as $T_P \coloneqq T_{e_k} \circ \cdots \circ T_{e_1}$.
The set of all simple $v$-$t$-Paths is denoted as $\paths_{v,t}$.
The \emph{earliest arrival time} at $t$ when starting at time $\theta$ in $v$ is then given by $l_{v,t}(\theta)\coloneqq \min_{P\in\paths_{v,t}} T_P(\theta)$.
Paths that attain this minimum are called \emph{shortest $v$-$t$-Path at time $\theta$}.

We call an edge $e=vw\in E$ \emph{active at time $\theta$}, if the condition $l_{v,t}(\theta) = l_{w,t}(T_e(\theta))$ holds true.
The set of active edges at time $\theta$ is collected in $E(\theta)$.
\todo[inline]{
    Isn't it disturbing, that these edges might not lie on a \textbf{simple} shortest path?
}

\subsection{Properties of FIFO Cost Functions}

To make sure that we can ignore non-simple paths in the definition of shortest paths, we exploit the FIFO property of the cost functions:

\begin{proposition}\label{prop:removing-cycles-in-fifo-graphs}
    Given cost functions $c:\R\rightarrow\R_{\geq0}^E$ following the FIFO rule, removing a cycle in any $v$-$t$-path $P$ does not increase the path's traversal time.

    More specifically, if $P=P_1\, C \, P_2$ is the concatenation of paths $P_1$, a cycle $C$ and another path $P_2$, then it holds that $
        T_P(\theta) \geq \left(
        T_{P_2} \circ T_{P_1} \right)(\theta)$ for all $\theta\in\R$.
\end{proposition}
\begin{proof}
The statement is a direct consequence of the monotonicity of $T_{P_2}$ and the fact that $T_C(\theta) \geq \theta$ holds for all $\theta\in\R$.
\end{proof}

\begin{proposition}\label{prop:characterization-arrival-functions}
    Let $c:\R\rightarrow\R_{>0}^E$ be a cost function following the FIFO rule.
    The vector $(l_{v,t})_{v\in V}$ of functions is the (pointwise) maximal solution of the following system of equations in the function-valued variables $(\tilde l_v: \R \rightarrow \R)_{v\in V}$:
    \[
        \tilde l_v(\theta) = \begin{cases}
            \theta, &\text{if $v = t$}, \\
            \min_{\substack{
                e=vw\in\outEdges{v}               
            }} \tilde{l}_w\left(
                T_e(\theta)
            \right), &\text{otherwise}.
        \end{cases}
    \]
\end{proposition}
\begin{proof}
    To see that $(l_{v,t})_{v\in V}$ is a solution of the system, we note that for $v = t$ we have $l_{v,t}(\theta) = \theta$ for all $\theta\in\R$.
    For $v\neq t$ let $e=vw\in\outEdges{v}$.
    Let $P$ be a shortest $w$-$t$-path at time $T_{e}(\theta)$.
    Let $P'\coloneqq e\circ P$ be the concatenation of $e$ and $P$.

    If $P'$ contains a cycle, then it must have been introduced by $e$ and $v$ must be visited a second time in $P'$.
    Removing this cycle gives a simple $v$-$t$-path $Q$ with 
    \[
        l_{v,t}(\theta) \leq T_{Q}(\theta)\leq T_{P'}(\theta) = l_{w,t}(T_{e}(\theta))
    \]
    by Proposition~\ref{prop:removing-cycles-in-fifo-graphs}.
    If $P'$ does not contain a cycle, we follow analogously
    \[
        l_{v,t}(\theta) \leq T_{P'}(\theta) = l_{w,t}(T_{e}(\theta)).
    \]
    This shows that $l_{v,t}(\theta)$ is a lower bound on $\{ l_{w,t}(
        T_{e}(\theta)
    ) \mid e=vw\in\outEdges{v}  \}$.
    Let $P = e_1\, \cdots\, e_k$ be a shortest $v$-$t$-path at time $\theta$ and let $e_1=vw$ and $P' = e_2\,\cdots\,e_k$.
    Furthermore, let $Q$ be a shortest $w$-$t$-path at time $T_{e_1}(\theta)$.
    Assume $T_{P}(\theta) > l_{w,t}(T_{e_1}(\theta))$.
    This implies
    \[
        T_{P}(\theta) > 
        T_{Q}(T_{e_1}(\theta)),
    \]
    such that removing any cycles in the path $e_1\circ Q$ would yield a strictly shorter path than $P$; a contradiction.

    We now have to show that $(l_{v,t})_{v\in V}$ is the maximal solution.
    That means that for any other solution $(\tilde l_v)_{v\in V}$ of the system of equations
     $\tilde l_{v}(\theta) \leq l_v(\theta)$ should hold for all $v\in V$ and $\theta\in\R$.
    Let $P = e_1 \,\cdots e_k$ be a shortest $v$-$t$-path at time $\theta$ and let $v_0 = v, v_k= t$ and $e_i = v_{i-1} v_i$ for $i=1,\dots,k$.
    Then by the system of equations we infer \[
        \tilde l_{v}(\theta) \leq \tilde l_{v_1}(T_{e_1}(\theta)) \leq \tilde l_{v_2,t}(T_{e_1\,e_2} (\theta)) \leq \cdots \leq \tilde l_{t}(T_P(\theta)) = T_P(\theta) = l_{v,t}(\theta).
    \]
\end{proof}

The following example illustrates why the system of equations above does not always have a unique solution.
The method used works in any cyclic graph: We build a self-confirming cycle proposing an earlier arrival time than actually possible.
\begin{example}
    A minimal example without loops consists of three nodes $V=\{ s, v, t \}$ with three edges $E=\{ st, sv, vs \}$ whose cost functions $c_e: \R \rightarrow \R_{\geq 0}$ follow the FIFO rule.
    We virtually limit the arrival time of $s$ at $0$ using $\tilde l_s (\theta) \coloneqq \min\{ T_{st}(\theta), 0 \}$.
    We define the arrival times of $v$ and $t$ as expected with $\tilde l_t(\theta) \coloneqq \theta$ and $\tilde l_v(\theta) \coloneqq \tilde l_s(T_{vs}(\theta))$.
    Obviously, the system of equations is satisfied for nodes $v$ and $t$.
    For node $s$, the equation reads \[
        \tilde l_s(\theta) = \min\left\{ \tilde l_t(T_{st}(\theta)), \tilde l_v(T_{sv}(\theta)) \right\}. 
    \]
    By inserting the definitions of $\tilde l_v$ and $\tilde l_t$, the right-hand side equates to \[
        \min\left\{ T_{st}(\theta), \tilde l_s(T_{vs}(T_{sv}(\theta))) \right\}
        = \min \left\{ T_{st}(\theta), T_{st} \Big( T_{vs}\big(T_{sv}(\theta)\big) \Big), 0 \right\}.
    \]
    Proposition~\ref{prop:removing-cycles-in-fifo-graphs} implies the inequality $T_{st}(\theta) \leq T_{st} ( T_{vs}(T_{sv}(\theta)))$, so that the right-hand side reduces to $\min\{ T_{st}(\theta), 0 \} = \tilde l_s(\theta)$.
    Obviously, this solution $(\tilde l_w)_{w\in V}$ is different to the actual arrival times $(l_w)_{w\in V}$: The arrival time when starting in $s$ at any positive time $\theta$ fulfills $l_s(\theta) = T_{st}(\theta) \geq \theta > 0 = \tilde l_s(\theta)$.
\end{example}

\subsection{Duality of Arrival and Departure Times}

Sometimes it is useful not to work on the earliest arrival time, but the latest possible departure time.
To enable this switch, we define a kind of inverse of a monotonically increasing function:

\newcommand{\IncCoercive}{\mathcal{F}}
\begin{definition}
    We define the function space 
    \[
        \IncCoercive \coloneqq \left\{ f:\R \rightarrow \R \mid \text{ $f$ is increasing and $ \lim_{\abs{x}\rightarrow\infty} \abs{f(x)} = \infty$} \right\}.
    \]
    The \emph{reversal of $f\in\IncCoercive$} is defined as 
    \[
        \minv{f}:\R \rightarrow \R,\ \theta\mapsto \sup \{ \xi\in\R \mid f(\xi) \leq \theta \}.
    \]
\end{definition}

We can interpret the reversal of $f$ in the following way:
If $f(\theta)$ is the earliest arrival time when departing at time $\theta$, then $\minv{f}(\theta)$ is the latest departure time for arriving before or at time $\theta$.

\begin{proposition}
    For $f,g\in \IncCoercive$ the following statements hold:
    \begin{enumerate}[label=(\roman*)]
        \item\label{prop:reversal-props:inner-operator} It holds that $\minv{f}\in\IncCoercive$, i.e. $\minv{f}$ is increasing and $\lim_{\abs{x}\rightarrow\infty} \abs{f(x)} = \infty$.
        \item\label{prop:reversal-props:continuous} If $f$ is continuous, then $\minv{f}(\theta) = \max \{ \xi\in\R \mid f(\xi) = \theta \}$ holds for all $\theta\in\R$ and $\minv{f}$ is strictly increasing.
        \item\label{prop:reversal-props:inverse} If $f$ is continuous and strictly increasing, then  $\minv{f}$ is the inverse of $f$.
        \item\label{prop:reversal-props:composition-minimum} It holds that $\minv{(g\circ f)} = \minv{f}\circ\minv{g}$ and $\minv{(\min\{f,g\})} = \max\{\minv{f}, \minv{g}\}$.
    \end{enumerate}
\end{proposition}
\begin{proof}
    \ref{prop:reversal-props:inner-operator}. 
    Let $\theta_1, \theta_2\in\R$ with $\theta_1 < \theta_2$.
    Then \begin{equation}\label{eq:reversal-props:increasing}
        \minv{f}(\theta_1) = \sup\{\xi\in\R \mid f(\xi) \leq \theta_1 \}
        \leq \sup\{\xi\in\R \mid f(\xi)\leq \theta_2 \} = \minv{f}(\theta_2)
    \end{equation}
    implies the monotonicity.
    From $\lim_{\abs{\theta}\rightarrow \infty} \abs{f(\theta)}$ and the monotonicity of $f$ we conclude \[
        \lim_{\abs{\theta}\rightarrow \infty} \abs{\minv{f}(\theta)}
        = \lim_{\abs{\theta}\rightarrow \infty} \abs{ \sup\{\xi\in\R \mid f(\xi)\leq \theta \} } = \infty.
    \]

    \ref{prop:reversal-props:continuous}.
    We note, that $\{ \xi\in\R \mid f(\xi) = \theta \}$ is non-empty, closed and bounded from above because of the condition $\lim_{\abs{\theta}\rightarrow \infty} \abs{f(\theta)} = \infty$ and the continuity and monotonicity of $f$.
    The reversal $\minv{f}$ is strictly increasing as the inequality~\eqref{eq:reversal-props:increasing} is strict for continuous $f$.
    
    \ref{prop:reversal-props:inverse}.
    Let $f$ be continuous and strictly increasing.
    Then it holds that 
    \[
        \minv{f}(\theta) = \max\{\xi\mid f(\xi) = \theta \} = f^{-1}(\theta).
    \]

    \ref{prop:reversal-props:composition-minimum}.
    After inserting the definition the first statement evaluated in $\theta$ becomes
    \[
        l\coloneqq \sup\left\{
            \xi \mid g(f(\xi)) \leq \theta
        \right\}
        =
        \sup\left\{
            \xi_f \mid f(\xi_f) \leq
            \sup\left\{ \xi_g \mid g(\xi_g) \leq \theta \right\}
            \right\}
         \eqqcolon r.
    \]
    Let $\xi_f\in\R$ fulfill $f(\xi_f) \leq \minv{g}(\theta)$.
    Then for all $\xi_g$ with $g(\xi_g) \leq \theta$ we have $f(\xi_f) \leq \xi_g$.
    Then because of the monotonicity of $g$ and $f$ we follow $g(f(\xi_f)) \leq g(\xi_g) \leq \theta$ which implies $l \geq r$.
    To see that $r\geq l$ holds, any $\xi\in\R$ with $g(f(\xi)) \leq \theta$ fulfills $f(\xi)\leq \minv{g}(\theta)$.

    The statement on the minimum of $f$ and $g$ evaluated in $\theta$ is of the form
    \[
        l\coloneqq \sup\{\xi \mid \min\{f(\xi),g(\xi)\} \leq \theta \}
        =
        \max\{
            \sup\{\xi \mid f(\xi)\leq \theta \},
            \sup\{\xi\mid g(\xi)\leq\theta\}
        \} \eqqcolon r.
    \]
    Let $\xi\in\R$ fulfill $\min\{f(\xi),g(\xi)\}\leq \theta$ and assume $f(\xi)\leq g(\xi)$ without loss of generality.
    Then it holds that $f(\xi) \leq \theta$ and therefore $\xi \leq \minv{f}(\theta)\leq r$ implying $l\leq r$.
    On the contrary, assume $\minv{f}(\theta)\leq \minv{g}(\theta)$ without loss of generality, so that $r = \minv{g}(\theta)$.
    Any $\xi\in\R$ with $g(\xi)\leq \theta$ fulfills $\min\{f(\xi),g(\xi)\}\leq g(\xi)\leq\theta$ and hence $l \geq r$ holds true.
\end{proof}

\begin{corollary}
    For a cost-function $c:\R\rightarrow\R_{\geq0}^E$, that follows the FIFO rule and that fulfills $\lim_{\theta\to-\infty} T_e(\theta) = -\infty$, the following statements hold true:
    \begin{enumerate}[label=(\roman*)]
        \item For all edges $e\in E$ we have $T_e \in \IncCoercive$.
        \item\label{prop:reversal-props:paths} For any path $P=e_1\,\cdots \,e_k$ it holds that $\minv{T_P} = \minv{T_{e_1}} \circ \cdots \circ \minv{T_{e_k}}$.
        \item For any node $v\in V$ it holds that $\minv{l_v} = \max_{P\in\paths_{v,t}} \minv{T_P}$.
    \end{enumerate}
\end{corollary}



\subsection{The Dynamic Dijkstra Algorithm}

The first algorithm we discuss is a simple modification of the Dijkstra Algorithm to determine the earliest arrival times $(l_{s,w}(\theta))_{w\in V'}$ at nodes of some subset $V'\subseteq V$ when departing from a source node $s$ at time $\theta$.
Here, only those nodes are relevant for us that are reachable from $s$ and that can reach $t$.
Moreover, we only need to determine the arrival times of nodes $w$ that can be reached before $t$, i.e. that fulfill $l_{s,w}(\theta) \leq l_{s,t}(\theta)$.
Hence, the set $V'$ consists of all nodes $w$ that lie on a path from $v$ to $t$ and fulfill $l_{s,w}(\theta) \leq l_{s,t}(\theta)$.

\begin{algorithm}[h]
\begin{minted}[mathescape, linenos]{python}
def dynamic_dijkstra(
    theta: float, source: Node, sink: Node, relevant_nodes: Set[Node],
    costs: List[Callable[[float], float]]
) -> Dict[Node, float]:
    arrival_times: Dict[Node, float] = {}
    queue: PriorityQueue[Node] = PriorityQueue([(source, theta)])
    while len(queue) > 0:
        arrival_time, v = queue.min_key(), queue.pop()
        arrival_times[v] = arrival_time
        if v == sink:
            break
        for e in v.outgoing_edges:
            w = e.node_to
            if w in arrival_times.keys() or w not in relevant_nodes:
                continue
            relaxation = arrival_time + costs[e.id](arrival_time)
            if not queue.contains(w):
                queue.push(w, relaxation)
            elif relaxation < queue.key_of(w):
                queue.decrease_key(w, relaxation)
    return arrival_times
\end{minted}
\caption{The Dynamic Dijkstra Algorithm}
\label{alg:dynamic-dijkstra}
\end{algorithm}

Adjusting the classical Dijkstra Algorithm to our setting yields the Dynamic Dijkstra Algorithm as depicted in Algorithm~\ref{alg:dynamic-dijkstra}.

At the heart of the algorithm operates a priority queue consisting of items together with a priority key associated with each item.
This queue has to support the operations \texttt{push(item, key)}, \texttt{min\_key()}, \texttt{pop()}, \texttt{decrease\_key(item, new\_key)} as well as \texttt{contains(item)}.
The operation \texttt{push(item, key)} adds the item \texttt{item} with priority \texttt{key} to the queue, \texttt{min\_key()} returns the minimum key of an item in the queue, \texttt{pop()} returns the item with minimum key and removes it from the queue, \texttt{contains(item)} returns whether \texttt{item} is contained in the queue and the operation \texttt{decrease\_key(item, new\_key)} replaces the priority key associated to the item \texttt{item} with \texttt{new\_key}.


This priority queue holds all discovered nodes where the priority key of a node $w$ is its currently suspected earliest arrival time, an upper bound on $l_{s,w}(\theta)$.


\begin{proposition}
    Given cost functions $c:\R \rightarrow \R_{\geq0}^E$ following the FIFO rule, the Dynamic Dijkstra Algorithm initiated on the source $v\in V$ and a reachable sink $t\in V$ computes the vector $(l_{s,w}(\theta))_{w\in V'}$ with \[
        V' = \{ w\in V \mid \text{$w$ lies on a $v$-$t$-path and $l_{s,w}(\theta) \leq l_{s,t}(\theta)$} \}.
    \]
\end{proposition}
\begin{proof}
    As an invariant for the algorithm we proof, that once a value of a node $w$ is written to \texttt{arrival\_times[w]}, this value coincides with $l_{s,w}(\theta)$.
    In the beginning this is clearly true as no values are written beforehand.
    In the first loop iteration $\texttt{v}$ is the source $s$ and its key $\theta= l_{v,v}(\theta)$ is written to \texttt{arrival\_times[v]}.
    Assume the loop invariant holds before entering loop body another time.
    As nodes are added at most once to the queue, we have $\texttt{v} \neq s$.
    Let $u$ be the any node in whose iteration $\texttt{v}$ was added to the queue or the key of $\texttt{v}$ in the queue was modified.
    The invariant implies $\texttt{arrival\_times[}u\texttt{]} = l_{\texttt{v},u}(\theta)$ and thus $\texttt{arrival\_time} = l_{s,u}(\theta) + c_{u\texttt{v}}(l_{s,u}(\theta)) = T_{u\texttt{v}}(l_{s,u}(\theta)) \geq l_{s,\texttt{v}}(\theta)$.

    Assume $\texttt{arrival\_time} > l_{s,\texttt{v}}(\theta)$ and let $P$ be a shortest $s$-$\texttt{v}$-path at time $\theta$.
    If all nodes of $P$ were available in \texttt{arrival\_times}, then so is the last one $u$ before $\texttt{v}$ in $P$ which would have set the key of \texttt{v} in its iteration to $T_{u\texttt{v}}(l_{s,u}(\theta)) = l_{s,\texttt{v}}(\theta)$.
    Let $u$ be the first node in $P$ that is not available in \texttt{arrival\_times}.
    Because $u$ cannot be the source $s$, the  predecessor $u'$ of $u$ in $P$ must have set the key of $u$ to at most $T_{u'u}(l_{s,u'}(\theta)) \leq T_P(\theta)$.
    As $\texttt{arrival\_time}>l_{s,\texttt{v}}(\theta) = l_P(\theta)$ the key of $u$ in the queue was smaller than the key of \texttt{v}, so the priority queue would have popped $u$ before \texttt{v}.
\end{proof}

A simple binary min-heap together with a lookup table was implemented to support the operations of the queue efficiently.
With this data structure, the worst case running time is logarithmic for the operations \texttt{push(item,key)}, \texttt{pop()} and \texttt{decrease\_key(item, new\_key)} and constant for the operations \texttt{min\_key()} and \texttt{contains(item)}.
Thus, the Dynamic Dijkstra Algorithm terminates with a running time of $\bigO( (\abs{V} + \abs{E}) \cdot \log \abs{V})$.


\subsection{Computing Active Outgoing Edges}

Given a FIFO cost function $c: \R\rightarrow\R_{\geq0}^E$ with a node $s$, a time $\theta\in\R$ and a sink $t$, we now want to compute the active outgoing edges $E(\theta) \cap \outEdges{s}$ of $s$, i.e. the edges $e=sw$ with 
\[
    l_{s,t}(\theta) = l_{w,t}(T_e(\theta)).
\]
Unfortunately, from the arrival times $(l_{s,w}(\theta))_{w\in V'}$ obtained by a simple run of the Dynamic Dijkstra Algorithm, we cannot determine all active edges:
Usually, the idea for determining all active edges would be to do a backward search starting from $t$ on edges $e=vw$ with $l_{s,w}(\theta) = T_e(l_{s,v}(\theta))$ and remember those edges as \emph{active} as they lie on a shortest $s$-$t$-path.
An implementation of this is shown in Algorithm~\ref{alg:backward-search}, where \texttt{arrivals} consists of the labels $(l_{s,w}(\theta))_{w\in V'}$ as returned by the Dynamic Dijkstra Algorithm.

\begin{algorithm}[h]
    \begin{minted}[mathescape, linenos]{python}
def backward_search(
    costs: List[Callable[[float], float]], arrivals: Dict[Node, float],
    source: Node, sink: Node
) -> Set[Edge]:
    active_edges = []
    queue: List[Node] = [sink]
    nodes_enqueued: Set[Node] = {sink}
    while len(queue) > 0:
        w = queue.pop()
        for e in w.incoming_edges:
            v = e.node_from
            if v not in arrivals.keys():
                continue
            if arrivals[v] + costs[e.id](arrivals[v]) <= arrivals[w]:
                if v == source:
                    active_edges.append(e)
                if v not in nodes_enqueued:
                    queue.append(v)
                    nodes_enqueued.add(v)
    return active_edges
    \end{minted}
    \caption{Backward Search}
    \label{alg:backward-search}
\end{algorithm}

However, there might be edges $e=vw$ with $l_{s,w}(\theta) < T_e(l_{s,v}(\theta))$ that lie on a shortest $s$-$t$-path at time $\theta$.
This might be the case if a bottleneck edge $e'$ closer to $t$ has an interval on which $T_{e'}$ is constant.

The following proposition proves that paths found using the described approach are in fact shortest paths.
Moreover, we show that for strong FIFO costs we find all shortest paths.
That means, for strong FIFO costs, the Backward Search Algorithm is correct.

\begin{proposition}
    Let $P=e_1\,\cdots\,e_k$ be a path with $e_i = v_{i-1}v_{i}$ and $v_0 = s, v_k= t$ and let $c:\R\rightarrow\R^E_{\geq0}$ be a cost function following the FIFO rule. Then
    \[ 
        \left(\forall i:\, T_{e_i}(l_{s,v_{i-1}}(\theta)) = l_{s,v_i}(\theta)\right)
        \implies
        T_P(\theta) = l_{s,t}(\theta).
    \]
    If $c$ follows the strong FIFO rule, the statements are equivalent.
\end{proposition}
\begin{proof}
    Assume $T_{e_i}(l_{s,v_{i-1}}(\theta)) = l_{s,v_i}(\theta)$ holds for all $i$.
    Then we have \begin{align*}
        T_P(\theta)
        &= T_P(l_{s,s}(\theta))
        = T_{e_2\,\cdots\,e_{k-1}}(l_{s,v_1}(\theta))
        = \cdots
        = l_{s,v_k}(\theta) = l_{s,t}(\theta).
    \end{align*}

    Let $c$ now follow the strong FIFO rule and assume there is some $i\in\{1,\dots, k\}$ with $T_{e_i}(l_{s,v_{i-1}}(\theta)) > l_{s,v_i}(\theta)$.
    Let $P'$ be a shortest $s$-$v_i$-path at time $\theta$.
    If we extend $P'$ with $e_{i+1}\,\cdots\,e_{k}$ we get an $s$-$t$-path $P''$ with \[
        l_{s,t}(\theta) \leq T_{P''}(\theta)
        = T_{e_{i+1}\,\cdots\,e_k}(l_{s,v_{i}}(\theta))
        < T_{e_{i+1}\,\cdots\,e_k}(T_{e_1\,\cdots\,e_i}(\theta))
        = T_P(\theta),
    \]
    by the strong monotonicity of $T_e$ for all $e\in E$.
\end{proof}

This leaves us with the question of how to find the rest of the active edges for general FIFO cost functions.
For the rest of this section, let $c:\R\rightarrow\R^E_{\geq0}$ follow the FIFO rule with $\lim_{\theta\to-\infty} T_e(\theta) = -\infty$ for all $e\in E$.
The idea we want to exploit now is that once we determined $l_{s,t}(\theta)$, we can do another run of the Dynamic Dijkstra Algorithm on the reverse graph $\rev{G} = (V, \rev{E})$ with $\rev{E}\coloneqq \{ \rev{e} = wv \mid e=vw\in E \}$ to compute the latest departure time to arrive before or at $l_{s,t}(\theta)$ at $t$.

We define the corresponding cost function $\tilde c$ as \[
    \tilde c : \R \rightarrow \R_{\geq0}^{\rev E}, \quad
    \tilde c_{\rev e}(\theta) \coloneqq - \minv{T_e}(-\theta) - \theta.
\]
Note, that $T_e  \geq \id_\R$ and by Proposition~\ref{prop:reversal-props}~\ref{prop:reversal-props:flip-id} we infer $\minv{T_e} \leq \id_\R$.
Therefore, we have $\minv{T_e}(-\theta) \leq -\theta$ and thus $\tilde c_{\rev e}(\theta) \geq 0$.

We denote the traversal times induced by $\tilde c$ by $\tilde T_{\rev e}$ and $\tilde T_{\rev P}$,where $P = \rev{e_k} \, \cdots \, \rev{e_1}$ for $P = e_1\,\cdots\, e_k$,
and the earliest arrival time due to $\tilde c$ as $\tilde l_{v,w}$.
The traversal times fulfill $\tilde T_e = -\minv{T_e}(-\theta)$ which is by Proposition~\ref{prop:reversal-props}~\ref{prop:reversal-props:continuous} strictly increasing.
Hence, $\tilde c$ follows the strong FIFO rule.
For a path $\rev{P} = \rev{e_k}\,\cdots\, \rev{e_1}$ the traversal time yields
\[
    \tilde T_{\rev P} (\theta)
    = \tilde T_{\rev{e_{k-1}}\,\cdots\,\rev{e_1}}(- \minv{T_{e_k}} ( - \theta ))
    = \tilde T_{\rev{e_{k-2}}\,\cdots\,\rev{e_1}}(-\minv{T_{e_{k-1}}} (\minv{T_{e_k}} ( - \theta ))
    = \cdots
    = - \minv{T_P}(-\theta)
\]
and the earliest arrival times are of the form
\[
    \tilde l_{v,w}(\theta)
    = \min_{P\in\paths_{w,v}} \tilde T_{\rev{P}}(\theta)
    = \min_{P\in\paths_{w,v}} - \minv{T_P}(-\theta)
    = - \max_{P\in\paths_{w,v}} \minv{T_P}(-\theta)
    = - \minv{l_{w,v}}(-\theta)
\]

With this setup, we can do a second run of the Dynamic Dijkstra Algorithm to obtain the vector \[
    (\tilde l_{t,w}( - l_{s,t}(\theta) ))_w = (- \minv{l_{t,w}}( l_{s,t}(\theta) ))_w.
\]
Using Lemma~\ref{lem:characterization-active-edges} we can now check whether an outgoing edge $sw\in\outEdges{s}$ of $s$ is active at time $\theta$ by evaluating $T_e(\theta) \leq \minv{l_{t,w}}( l_{s,t}(\theta))$.

\begin{algorithm}[h]
    \begin{minted}[mathescape, linenos]{python}
def get_active_edges(
  costs: List[PiecewiseLinear], theta: float, source: Node, sink: Node,
  relevant_nodes: Set[Node], graph: DirectedGraph, strong_fifo: bool
) -> Set[Edge]:
  if len(source.outgoing_edges) <= 1:
    return source.outgoing_edges
  arrivals = dynamic_dijkstra(
    theta, source, sink, relevant_nodes, costs
  )
  if strong_fifo:
    return backward_search(costs, arrivals, source, sink)
  else: # Second run of Dijkstra on the reverse graph.
    graph.reverse()
    traversals = [cost + identity for cost in costs]
    new_costs: List[Callable[[float], float]] = [
      lambda t: -trav.reversal(-t) - t for trav in traversals
    ]
    neg_departures = dynamic_dijkstra(
      -arrivals[sink], sink, source, relevant_nodes, new_costs
    )
    graph.reverse()
    return [
      e for e in source.outgoing_edges
      if traversals[e.id](theta) <= -neg_departures[e.node_to]
    ]
  \end{minted}
  \caption{Calculating Active Edges}
  \label{alg:calculate-active-edges}
\end{algorithm}


To implement this algorithm, we have to restrict ourselves to cost functions where the reversal of $T_e$ can be evaluated easily.
This is the case, for example, if $c_e$ are piecewise linear functions.
The resulting algorithm for this class of functions can be seen in Algorithm~\ref{alg:calculate-active-edges}.
Here, the flag \texttt{strong\_fifo} is expected to be set only for cost functions following the strong FIFO rule.
For these type of functions, we use the simpler backward search; for  general FIFO functions, we use the approach described above.
Other than that, if $s$ has only up to one outgoing edge, we simply return $\outEdges{s}$.


\subsection{Computing the Arrival Functions}

Often, it is useful not only to compute active edges at some fixed point in time, but to have the earliest arrival functions $(l_{w,t})_{w\in V'}$ at some sink available as functions over time.

A very simple but also quite expensive method of computing these functions is a modification of the Bellman-Ford algorithm. \todo{Add cite}
It uses the representation found in Proposition~\ref{prop:characterization-arrival-functions}.
This means, we want to find the maximal solution of the system of equations \[
    \tilde l_v(\theta) = \begin{cases}
        \theta, &\text{if $v = t$}, \\
        \min_{\substack{
            e=vw\in\outEdges{v}               
        }} \tilde{l}_w\left(
            T_e(\theta)
        \right), &\text{otherwise}.
    \end{cases}
\]
Hence, the idea is to initialize all functions with $\tilde l_v(\theta) \coloneqq \infty$ for $v\neq t$ and $\tilde l_t(\theta) \coloneqq \theta$ and decrease the functions (pointwise) using the operation $\tilde l_v \coloneqq \min_{vw\in\outEdges{v}} \tilde l_w \circ T_{vw}$ until no further changes are made and the equations are fulfilled for all $v\in V$.

More specifically, if some function $\tilde l_w$ changes, then all nodes $v$ with an edge $vw$ leading to $w$ might need to be adjusted as well using the operation $\tilde l_v \coloneqq \min\{ \tilde l_v, \tilde l_w \circ T_{vw} \}$.
Therefore, we need some operations on the class of functions operated on to formulate the algorithm:
To calculate the traversal times $T_e = c_e + \id_\R$ we need pointwise addition and a representation of the identity function; for updates of the functions the pointwise minimum and the composition of functions have to be implemented.
To detect changes we also need to be able to identify whether one function is everywhere smaller or equal to some other function. 
Instead of representing the constant function with value $\infty$, we can simply only add a function once it gets its first update.

\begin{algorithm}
    \begin{minted}[mathescape,
        linenos,
        numbersep=5pt,
        framesep=5mm]{python}
def dynamic_bellman_ford(
    sink: Node, costs: List[PiecewiseLinear], relevant_nodes: Set[Node],
    theta: float
) -> Dict[Node, PiecewiseLinear]:
    arrivals: Dict[Node, PiecewiseLinear] = { sink: identity }
    traversals = [cost + identity for cost in costs]
    changed_nodes = {sink}
    while len(changed_nodes) > 0:
        change_detected = {}
        for w in changed_nodes:
            for e in w.incoming_edges:
                v = e.node_from
                if v not in relevant_nodes:
                    continue
                relaxation = arrivals[w].compose(traversals[e.id])
                if v not in arrivals.keys():
                    change_detected.add(v)
                    arrivals[v] = relaxation
                elif not arrivals[v] <= relaxation:
                    change_detected.add(v)
                    arrivals[v] = arrivals[v].minimum(relaxation)
        changed_nodes = change_detected
    return arrivals
    \end{minted}
    \caption{Dynamic Bellman-Ford Algorithm}
    \label{alg:dynamic-bellman-ford}
\end{algorithm}

All the necessary operations explained above have been created for piecewise linear functions.
The resulting procedure is shown in Algorithm~\ref{alg:dynamic-bellman-ford}.
Here, \code{cost + identity} is the piecewise linear function representing the sum of \code{cost} and \code{identity}, the expression \code{arrivals[w].compose(traversals[e.id])} computes the piecewise linear function representing $\code{arrivals[w]}\circ \code{traversals[e.id]}$.
The decision whether the inequality $\code{arrivals[v]}(\theta) \leq \code{relaxation}(\theta)$ holds for all $\theta \in \R$ is expressed by the term \code{arrivals[v] <= relaxation}. 
Finally, the pointwise minimum of the functions \code{arrivals[v]} and \code{relaxation} is computed by \code{arrivals[v].minimum(relaxation)}.