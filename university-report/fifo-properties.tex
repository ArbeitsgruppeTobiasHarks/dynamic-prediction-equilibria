\subsection{Properties of FIFO Cost Functions}

To make sure that we can ignore non-simple paths in the definition of shortest paths, we exploit the FIFO property of the cost functions:

\begin{proposition}\label{prop:removing-cycles-in-fifo-graphs}
    Given cost functions $c:\R\rightarrow\R_{\geq0}^E$ following the FIFO rule, removing a cycle in any $v$-$t$-path $P$ does not increase the path's traversal time.

    More specifically, if $P=P_1\, C \, P_2$ is the concatenation of paths $P_1$, a cycle $C$ and another path $P_2$, then it holds that $
        T_P(\theta) \geq \left(
        T_{P_2} \circ T_{P_1} \right)(\theta)$ for all $\theta\in\R$.
\end{proposition}
\begin{proof}
The statement is a direct consequence of the monotonicity of $T_{P_2}$ and the fact that $T_C(\theta) \geq \theta$ holds for all $\theta\in\R$.
\end{proof}

\begin{proposition}\label{prop:characterization-arrival-functions}
    Let $c:\R\rightarrow\R_{>0}^E$ be a cost function following the FIFO rule.
    The vector $(l_{v,t})_{v\in V}$ of functions is the (pointwise) maximal solution of the following system of equations in the function-valued variables $(\tilde l_v: \R \rightarrow \R)_{v\in V}$:
    \[
        \tilde l_v(\theta) = \begin{cases}
            \theta, &\text{if $v = t$}, \\
            \min_{\substack{
                e=vw\in\outEdges{v}               
            }} \tilde{l}_w\left(
                T_e(\theta)
            \right), &\text{otherwise}.
        \end{cases}
    \]
\end{proposition}
\begin{proof}
    To see that $(l_{v,t})_{v\in V}$ is a solution of the system, we note that for $v = t$ we have $l_{v,t}(\theta) = \theta$ for all $\theta\in\R$.
    For $v\neq t$ let $e=vw\in\outEdges{v}$.
    Let $P$ be a shortest $w$-$t$-path at time $T_{e}(\theta)$.
    Let $P'\coloneqq e\circ P$ be the concatenation of $e$ and $P$.

    If $P'$ contains a cycle, then it must have been introduced by $e$ and $v$ must be visited a second time in $P'$.
    Removing this cycle gives a simple $v$-$t$-path $Q$ with 
    \[
        l_{v,t}(\theta) \leq T_{Q}(\theta)\leq T_{P'}(\theta) = l_{w,t}(T_{e}(\theta))
    \]
    by Proposition~\ref{prop:removing-cycles-in-fifo-graphs}.
    If $P'$ does not contain a cycle, we follow analogously
    \[
        l_{v,t}(\theta) \leq T_{P'}(\theta) = l_{w,t}(T_{e}(\theta)).
    \]
    This shows that $l_{v,t}(\theta)$ is a lower bound on $\{ l_{w,t}(
        T_{e}(\theta)
    ) \mid e=vw\in\outEdges{v}  \}$.
    Let $P = e_1\, \cdots\, e_k$ be a shortest $v$-$t$-path at time $\theta$ and let $e_1=vw$ and $P' = e_2\,\cdots\,e_k$.
    Furthermore, let $Q$ be a shortest $w$-$t$-path at time $T_{e_1}(\theta)$.
    Assume $T_{P}(\theta) > l_{w,t}(T_{e_1}(\theta))$.
    This implies
    \[
        T_{P}(\theta) > 
        T_{Q}(T_{e_1}(\theta)),
    \]
    such that removing any cycles in the path $e_1\circ Q$ would yield a strictly shorter path than $P$; a contradiction.

    We now have to show that $(l_{v,t})_{v\in V}$ is the maximal solution.
    That means that for any other solution $(\tilde l_v)_{v\in V}$ of the system of equations
     $\tilde l_{v}(\theta) \leq l_v(\theta)$ should hold for all $v\in V$ and $\theta\in\R$.
    Let $P = e_1 \,\cdots e_k$ be a shortest $v$-$t$-path at time $\theta$ and let $v_0 = v, v_k= t$ and $e_i = v_{i-1} v_i$ for $i=1,\dots,k$.
    Then by the system of equations we infer \[
        \tilde l_{v}(\theta) \leq \tilde l_{v_1}(T_{e_1}(\theta)) \leq \tilde l_{v_2,t}(T_{e_1\,e_2} (\theta)) \leq \cdots \leq \tilde l_{t}(T_P(\theta)) = T_P(\theta) = l_{v,t}(\theta).
    \]
\end{proof}

The following example illustrates why the system of equations above does not always have a unique solution.
The method used works in any cyclic graph: We build a self-confirming cycle proposing an earlier arrival time than actually possible.
\begin{example}
    A minimal example without loops consists of three nodes $V=\{ s, v, t \}$ with three edges $E=\{ st, sv, vs \}$ whose cost functions $c_e: \R \rightarrow \R_{\geq 0}$ follow the FIFO rule.
    We virtually limit the arrival time of $s$ at $0$ using $\tilde l_s (\theta) \coloneqq \min\{ T_{st}(\theta), 0 \}$.
    We define the arrival times of $v$ and $t$ as expected with $\tilde l_t(\theta) \coloneqq \theta$ and $\tilde l_v(\theta) \coloneqq \tilde l_s(T_{vs}(\theta))$.
    Obviously, the system of equations is satisfied for nodes $v$ and $t$.
    For node $s$, the equation reads \[
        \tilde l_s(\theta) = \min\left\{ \tilde l_t(T_{st}(\theta)), \tilde l_v(T_{sv}(\theta)) \right\}. 
    \]
    By inserting the definitions of $\tilde l_v$ and $\tilde l_t$, the right-hand side equates to \[
        \min\left\{ T_{st}(\theta), \tilde l_s(T_{vs}(T_{sv}(\theta))) \right\}
        = \min \left\{ T_{st}(\theta), T_{st} \Big( T_{vs}\big(T_{sv}(\theta)\big) \Big), 0 \right\}.
    \]
    Proposition~\ref{prop:removing-cycles-in-fifo-graphs} implies the inequality $T_{st}(\theta) \leq T_{st} ( T_{vs}(T_{sv}(\theta)))$, so that the right-hand side reduces to $\min\{ T_{st}(\theta), 0 \} = \tilde l_s(\theta)$.
    Obviously, this solution $(\tilde l_w)_{w\in V}$ is different to the actual arrival times $(l_w)_{w\in V}$: The arrival time when starting in $s$ at any positive time $\theta$ fulfills $l_s(\theta) = T_{st}(\theta) \geq \theta > 0 = \tilde l_s(\theta)$.
\end{example}