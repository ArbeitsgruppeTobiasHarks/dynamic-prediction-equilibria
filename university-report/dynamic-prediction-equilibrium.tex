\subsection{Dynamic Prediction Equilibrium}\label{subsec-dynamic-prediction-equilibrium}

To enable a game theoretic perspective on dynamic flows, the infinitesimally small particles are interpreted as agents belonging to a commodity $i\in I$.
An agent gets generated at some time at the source node $s_i$.
Each time an agent arrives at a node, it can update its route based on historical traffic data of the network.
More specifically, each commodity has its own prediction method to predict future queue lengths on all edges of the network.
Based on the predicted queue lengths an agent now determines a time-dependent shortest path starting from his current position to $t_i$.

More formally, for a commodity $i\in I$ the \emph{predicted queue length} of an edge $e$ at time $\theta$ as predicted at time $\bar\theta$ based on the historical queue lengths $q=(q_e)_{e\in E}$ is given by $\predq_{i,e}(\theta, \bar\theta, q)$.
Aside from the predicted queue lengths all other predicted values will be denoted with a hat. 
This means, each commodity $i$ has a set of predictor functions $(\predq_{i,e})_{e\in E}$ each of which has the following the signature:
\[
    \predq_{i,e} : \R_{\geq 0} \times \R_{\geq 0} \times \contFcts{\R_{\geq 0}}{\R_{\geq 0}}^E \rightarrow \R_{\geq0},
\]
where $\contFcts{A}{B}$ denotes the set of all continuous functions from $A$ to $B$.

Based on the predicted queue lengths we can also determine \emph{predicted queue costs} by \[
    \predc_{i,e}(\theta, \bar\theta, q) \coloneqq \transit_e + \frac{\predq_{i,e}(\theta, \bar\theta, q)}{\capa_e}.
\]
The \emph{predicted traversal time of an edge $e=vw$} is the predicted arrival time at $w$ when entering edge $e$ at time $\theta$ defined as $\predT_{i,e}(\theta, \bar\theta, q) \coloneqq \theta + \predc(\theta, \bar\theta, q)$.
The \emph{predicted traversal time of a path $P=e_1\,e_2\,\cdots\,e_k$} is thus defined as \[
    \predT_{i,P}(\theta,\bar\theta, q) \coloneqq \left(
        \predT_{i,e_k}(\emptyArg, \bar\theta, q) \circ \cdots \circ \predT_{i,e_1}(\emptyArg, \bar\theta, q)
    \right) (\theta).
\]

For all nodes $v$ that can reach $t_i$, we denote the set of all simple $v$-$t_i$-paths as $\paths_{v,t_i}$.
We can then determine the \emph{predicted earliest arrival time} at $t_i$ when starting in $v$ at time $\theta$ by
\[
    \predl_{v,i}(\theta, \bar\theta, q) \coloneqq \min_{P \in \paths_{v,t_i}} \predT_{i, P}(\theta,\bar\theta, q).
\]
We call a path that attains this minimum a \emph{$\bar\theta$-predicted shortest path at time $\theta$}.
\todo[inline]{
    We will later see, why we can restrict ourselves to simple paths.
    (Choosing $\transit_e$ to be strictly positive is not enough. We need to have "FIFO-costs".)
}


\begin{proposition}
    For all $\bar\theta\in \R,q\in\contFcts{\R_{\geq0}}{\R_{\geq0}}$, the vector $(\predl_{v,i}(\emptyArg,\bar\theta, q))_{v\in V_i}$ of functions is the unique solution of the following system of equations in the function-valued variables $(l_v)_{v\in V_i}$:
    \[
        l_v(\theta) = \begin{cases}
            \theta, &\text{if $v = t_i$}, \\
            \min_{
                e=vw\in\outEdges{v}\atop
                w \in V_i                
            } l_w\left(
                \predT_{i, e}(\theta, \bar\theta, q)
            \right), &\text{otherwise}.
        \end{cases}
    \]
\end{proposition}
\begin{proof}
    \todo[inline]{Todo.}
\end{proof}