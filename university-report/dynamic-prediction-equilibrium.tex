\subsection{Dynamic Prediction Equilibrium}\label{subsec-dynamic-prediction-equilibrium}

To enable a game theoretic perspective on dynamic flows, the infinitesimally small particles are interpreted as agents belonging to a commodity $i\in I$.
An agent gets generated at some time at the source node $s_i$.
Each time an agent arrives at a node, it can update its route based on historical traffic data of the network.
More specifically, each commodity has its own prediction method to predict future queue lengths on all edges of the network.
Based on the predicted queue lengths an agent now determines a time-dependent shortest path starting from his current position to $t_i$.

More formally, for a commodity $i\in I$ the \emph{predicted queue length} of an edge $e$ at time $\theta$ as predicted at time $\bar\theta$ based on the historical queue lengths $q=(q_e)_{e\in E}$ is given by $\predq_{i,e}(\theta, \bar\theta, q)$.
Aside from the predicted queue lengths all other predicted values will be denoted with a hat. 
This means, each commodity $i$ has a set of predictor functions $(\predq_{i,e})_{e\in E}$ each of which has the following the signature:
\[
    \predq_{i,e} : \R \times \R \times \QueueFct^E \rightarrow \R_{\geq0},
\]
where $\QueueFct\coloneqq \contFcts{\R}{\R_{\geq0}}$ denotes the set of all continuous functions from $\R$ to $\R_{\geq0}$.

Based on the predicted queue lengths we can also determine \emph{predicted edge costs} by \[
    \predc_{i,e}(\theta, \bar\theta, q) \coloneqq \transit_e + \frac{\predq_{i,e}(\theta, \bar\theta, q)}{\capa_e}.
\]
The \emph{predicted traversal time of an edge $e=vw$} is the predicted arrival time at $w$ when entering edge $e$ at time $\theta$ defined as $\predT_{i,e}(\theta, \bar\theta, q) \coloneqq \theta + \predc(\theta, \bar\theta, q)$.
The \emph{predicted traversal time of a path $P=e_1\,e_2\,\cdots\,e_k$} is thus defined as \[
    \predT_{i,P}(\theta,\bar\theta, q) \coloneqq \left(
        \predT_{i,e_k}(\emptyArg, \bar\theta, q) \circ \cdots \circ \predT_{i,e_1}(\emptyArg, \bar\theta, q)
    \right) (\theta).
\]

For all nodes $v$ that can reach $t_i$, we denote the set of all simple $v$-$t_i$-paths as $\paths_{v,t_i}$.
We can then determine the \emph{predicted earliest arrival time} at $t_i$ when starting in $v$ at time $\theta$ by
\[
    \predl_{i,v}(\theta, \bar\theta, q) \coloneqq \min_{P \in \paths_{v,t_i}} \predT_{i, P}(\theta,\bar\theta, q).
\]
We call a path that attains this minimum a \emph{$\bar\theta$-predicted shortest $v$-$t_i$-path at time $\theta$}.
Before we continue, we want to make sure, that this definition of shortest-paths is sensible:
We can only restrict ourselves to simple paths, if waiting at a node, or traveling through a short cycle, is never helpful.
This can be realized by assuming the FIFO-property of predictors:
\begin{definition}
    A set of predictors $(\predq_{i,e})_{e\in E}$ is \emph{FIFO-compatible} if $(\predc_{i,e}(\emptyArg, \bar\theta, q))_{e\in E}$ follows the FIFO-rule for all $\bar\theta\in\R_{\geq0}, q\in\QueueFct^E$.
\end{definition}

We recall, that for FIFO-compatible predictors the results from Section~\ref{sec:fifo-costs} apply, and we have $\predl_{i,v}(\theta, \bar\theta, q) \leq \predl_{i,w}{(
    \predT_{i, e}(\theta, \bar\theta, q)
)}$
for all $v\in V_i$ and $e=vw\in E$.

\begin{definition}
An edge $e=vw$ is called \emph{active} for commodity $i\in I$ at time $\theta\in\R$ as predicted at time~$\bar\theta$, if the equation \[
    \predl_{i,v}(\theta, \bar\theta, q) = \predl_{i,w}{\left(
        \predT_{i, e}(\theta, \bar\theta, q)
    \right)}
\]
holds true.
All edges for commodity $i$ that are active at time $\theta$ as predicted at $\bar\theta$ are collected in the set $\predE_i(\theta, \bar\theta, q)$. 
\end{definition}

We can now define an equilibrium based on these predictors, in which all agents only take shortest paths at all times based on their current prediction.
\begin{definition}
    A pair $(\predq, f)$ of a set of predictors $(\predq_{i,e})_{i\in I, e\in E}$ and a dynamic flow $f$ is a \emph{partial dynamic prediction equilibrium up to time $H\in\R\cup\{\infty\}$} if $f$ is feasible up to time $H$ and for all $e\in E, i\in I$ and $\theta < H$ it holds that
    \[
        \infl_{i,e}(\theta) > 0 \implies e \in \predE_i(\theta, \theta, q).
    \]
    For $H = \infty$ the pair $(\predq, f)$ is called \emph{dynamic prediction equilibrium (DPE)}. 
    Moreover, we call $f$ a \emph{dynamic prediction flow with respect to the predictor $\predq$}. 

\end{definition}

In other words, a pair of a set of predictors and a flow is a DPE, if whenever flow enters an edge at some time $\theta$, then the prediction of that commodity evaluated at the same time $\theta$ says, that the edge lies on a dynamic shortest path to $t_i$.

DPEs have been introduced in~\cite{mainpaper}.
For the proof of the existence of DPEs the following two properties are important.

\begin{definition}
    A predictor $\predq_{e}$  is called \emph{oblivious},
if the following condition holds
    \[
    \forall \theta,\bar\theta, q,q'\colon\quad
    \restrictUpTo{q}{\bar\theta} = \restrictUpTo{q'}{\bar\theta}
    \implies
    \predq_{i,e}(\theta;\bar\theta;q)=\predq_{i,e}(\theta;\bar\theta;q'),
    \]
    where $\restrictUpTo{q}{\bar\theta}$ denotes the restriction of the function $q: \R_{\geq 0} \times E \to \R_{\geq 0}$ to $[0,\bar\theta] \times E$.
\end{definition}
\begin{definition}
    A predictor $\predq_{e}$ is \emph{continuous}, if the mapping 
    \[
        \predq_{e}: \R\times \R\times \QueueFct^E \rightarrow \R_{\geq0}
    \] with respect to the product topology on the left side and the topology induced by the uniform norm on $\QueueFct$ is continuous.
\end{definition}

\begin{theorem}[{\cite[Theorem~10]{mainpaper}}]\label{thm:dpe-existence}
    For any network with a finite set of commodities, each associated with a locally integrable, bounded network inflow rate and oblivious, continuous, and FIFO-compatible predictors $\predq_{i,e}$ there exists a dynamic prediction flow with respect to $\predq$. 
\end{theorem}

\subsection{Applied Predictors}\label{sec:applied-predictors}

In this section we introduce several predictors that have been used throughout the project.
We begin with very primitive predictors and enhance them step by step.


The \emph{Zero-Predictor} $\predq^\predZ$ predicts that no queues occur at all, i.e.\[
    \predq^\predZ_{e}(\theta, \bar\theta, q) \coloneqq 0
\]
for all $e\in E$, $\theta,\bar\theta\in\R$ and $q\in \QueueFct^E$.
This induces the constant edge costs $\predc^\predZ_e(\emptyArg, \emptyArg, \emptyArg)\equiv \transit_e$ for all edges $e\in E$.

Another approach is the so-called \emph{constant predictor} $\predq^\predC$ which predicts, that the queue lengths simply stay constant beginning from the prediction time, i.e.
\[
    \predq^\predC_{e}(\theta, \bar\theta, q) \coloneqq q_e(\bar\theta)
\]
for all $e\in E$, $\theta,\bar\theta\in\R$ and $q\in \QueueFct^E$.
Here, the edge costs are constant in the argument $\theta$ with $\predc^\predZ_e(\emptyArg, \bar\theta, q)\equiv \transit_e + q_e(\bar\theta)/\capa_e$ for all edges $e\in E$.
If all commodities use the constant predictor, the dynamic flow of a DPE is the same as an instantaneous dynamic equilibrium (IDE) as introduced in~\cite[Definition~2.1]{Graf2020}.

A more sophisticated variant is the \emph{linear predictor} $\predq^\predL$ which predicts that the queue length extends with the current derivative up to some prediction horizon $H\in \R_{>0} \cup \{ \infty \}$.
Starting from $H$ we assume a constant course.
In formulas, this can be written as \[
    \predq^\predL_e(\theta,\bar\theta,q) \coloneqq \left( q_e(\bar\theta) + \partial_- q_e(\bar\theta) \cdot \min\{\theta - \bar\theta, H \} \right)^+,
\]
where $(x)^+\coloneqq \max\{ x, 0 \}$ denotes the positive part of $x\in\R$.

As the left derivative of $q_e$ might not be continuous, the linear predictor does not qualify for the requirements of Theorem~\ref{thm:dpe-existence}.
The following predictor eliminates this problem by applying a regularization.
The \emph{regularized linear predictor} $\predq^\predRL$ uses a rolling average of the past gradient with rolling horizon $\delta>0$, which yields
\[
\hat q_{i,e}^{\text{RL}}(\theta, \bar\theta, q) \coloneqq
  \Big( q_e(\bar\theta) + \frac{q_e(\bar\theta) - q_e(\bar\theta - \delta)}{\delta} \cdot \min\{ \theta - \bar\theta, H \} \Big)^+.
\]

Furthermore, we introduce a machine learning predictor $\predq^\predML$ using a simple linear regression model.
More specifically, for an edge $e=vw$ we use $k_p$ samples of past queue lengths of surrounding edges $N(e) = \inEdges{v} \cup \{ e \} \cup\outEdges{w}$ to predict $k_f$ samples of the future queue length of $e$.
The samples are equidistant with step length $\delta > 0$.
For each edge $e$, we learn coefficient matrices $W^{e'}\in\R^{k_p\times k_f}$ for $e'\in N(e)$ as well as biases $\beta\in\R^{k_f}$ to compute interpolation points \[
    \predq^\predML_e\left(\bar\theta + j\cdot\delta,\bar\theta,q \right) \coloneqq 
    \left(
    \sum_{e'\in N(e)} \sum_{i = 1}^{k_p} w^{e'}_{i, j} \cdot {q_{e'}}{\left( 
        \bar\theta - (i-1)\cdot\delta
    \right)}
    + \beta_j
    \right)^+
\]
for $j\in\{1,\dots,k_f\}$ and $i\in\{1,\dots,k_p\}$.
The function $\predq^\predML_e(\emptyArg, \bar\theta, q)$ linearly interpolates these points and is extended constantly with the value $\predq^\predML_e(\bar\theta + k_f\cdot\delta,\bar\theta,q )$ starting from $\bar\theta + k_f\cdot \delta$.
As the above predictor is not necessarily FIFO-compatible per se, we manually limit the gradient of $\predq^\predML_e(\emptyArg, \bar\theta, q)$ such that it never subceeds $-\capa_e$.
More details on this method are described in Section~\ref{subsec:implementation-predictors}.

Finally, we introduce the \emph{perfect predictor} $\predq^\predP$ which predicts the queues as they will occur with \[
    \predq^\predP_e(\theta,\bar\theta,q) \coloneqq q_e(\theta).
\]
As this is a non-oblivious predictor, it is not possible to embed it into a simulation nor does it fulfill the constraints of Theorem~\ref{thm:dpe-existence}.
Nevertheless, the flow of a DPE equilibrium, in which all commodities use the perfect predictor, is the same as a dynamic equilibrium as analyzed in~\cite[Definition~2]{Cominetti2015}.
In this scenario particles can exactly predict the traffic conditions at any future time and choose a shortest path already when starting at the source node.
