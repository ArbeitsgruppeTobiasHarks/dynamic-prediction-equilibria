\documentclass[titlepage]{scrartcl}

\usepackage{amsmath}
\usepackage{amssymb}
\usepackage{amsthm}
\usepackage{hyperref}
\usepackage{mathtools}
\usepackage{todonotes}

\newtheorem{theorem}{Theorem}[section]
\theoremstyle{definition}
\newtheorem{definition}[theorem]{Definition}


\author{Michael Markl}
\title{Computation of Dynamic~Prediction~Equilibria}
\subtitle{A software project as part of the M.\,Sc. Mathematics at~the~University~of~Augsburg.}

\newcommand{\R}{\mathbb{R}}
\newcommand{\diff}{\,\textnormal{d}}

% Flow-specifics
\newcommand{\capa}{\nu}
\newcommand{\transit}{\tau}
\newcommand{\infl}[1][f]{{#1}^+}
\newcommand{\outfl}[1][f]{{#1}^-}
\newcommand{\qulen}{q}

\begin{document}
    \maketitle

    \newpage
    \tableofcontents

    \newpage
    \section{Introduction}

    The computation of dynamic flows is up to today a hard challenge.

    \newpage
    \section{Equilibrium Flows}

    We begin by defining the model of dynamic flows, also called flows over time: Here, Vickrey's fluid queuing model is used. \todo{Add cite}
    The traffic network is represented as a directed Graph $(V, E)$ with a finite set of nodes $V$ and a finite set of edges $E$.
    Although, we allow parallel edges, we often write $e=(v,w)\in E$ for a directed edge $e$ from node $v$ to node $w$.
    Each edge $e$ has a positive rate capacity $\capa_e > 0$ and a positive transit time $\transit_e > 0$.
    The rate capacity bounds the amount flow an edge can transfer at a time, which can be imagined as the width of a conveyor belt.
    The transit time on the other is the time the conveyor belt needs to transfer particles from its beginning to its end.


    
    \begin{definition}
        A \emph{dynamic flow} is a pair $(\infl, \outfl)$ of families of locally integrable functions with $\infl_e, \outfl_e : \R \rightarrow \R_{\geq 0}$ with $\infl_e(\theta) = \outfl_e(\theta) = 0$ for $\theta \leq 0$ and  $e\in E$.
        Here, $\infl_e(\theta)$ is called the \emph{inflow rate} of edge $e$ at time $\theta$ whereas $\outfl_e(\theta)$ describes the \emph{outflow rate} of $e$ at time $\theta$.

        Given a dynamic flow, we can define the \emph{accumulative inflow and outflow} as
        \[
            F^+_e(\theta) \coloneqq \int_0^\theta \infl_e(t) \diff t \quad\text{and}\quad  F^-_e(\theta) \coloneqq \int_0^\theta \outfl_e(t) \diff t 
        \]
        respectively.
        The \emph{queue length} of an edge $e$ at time $\theta$ can then be determined as $\qulen_e(\theta) \coloneqq F^+_e(\theta) - F^-_e(\theta + \transit_e)$.
        

        A dynamic flow is called \emph{feasible}, if it fulfills the following conditions:
        \begin{enumerate}
            \item 
        \end{enumerate}
        
    \end{definition}
        

    \newpage
    \section{Single-Commodity Flows}

    \subsection{Piecewise Linear Functions}



\end{document}