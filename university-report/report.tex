\documentclass[titlepage]{scrartcl}

\author{Michael Markl}
\title{Computation of Dynamic~Prediction~Equilibria}
\subtitle{A software project as part of the M.\,Sc. Mathematics at~the~University~of~Augsburg.}

\newcommand{\capa}{\nu}
\newcommand{\transit}{\tau}

\begin{document}
    \maketitle

    \newpage
    \tableofcontents

    \newpage
    \section{Introduction}

    The computation of dynamic flows is up to today a hard challenge.

    \newpage
    \section{Equilibrium Flows}

    We begin by defining the model of dynamic flows, also called flows over time, which is used here.
    The traffic network is represented as a directed Graph $(V, E)$ with a finite set of nodes $V$ and a finite set of edges $E$.
    Although, we allow parallel edges, we often write $e=(v,w)\in E$ for a directed edge $e$ from node $v$ to node $w$.
    Each edge $e$ has a positive rate capacity $\capa_e > 0$ and a positive transit time $\transit_e > 0$.
    The rate capacity bounds the amount flow an edge can transfer at a time, which can be imagined as the width of a conveyor belt.
    The transit time on the other is the time the conveyor belt needs to transfer particles from its beginning to its end.

    %We use Vickrey's fluid queuing model in which agents are described as infinitesimally small particles.
    
    

    \newpage
    \section{Single-Commodity Flows}

    \subsection{Piecewise Linear Functions}



\end{document}